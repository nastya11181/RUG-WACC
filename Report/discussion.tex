\section{Discussion}

In this chapter we review and reflect on the project.

In the beginning of the project we started with developing an UI. We used a nginx webserver, but we struggled to put it in a docker container and connect it to the back-end. It took us some time to figure it out how exactly to connect these parts. We managed to retrieve data from the database and to present it in the UI. We chose to use Angular as a framework for our UI.

It took us some time to set up the back-end with Play and Scala. Since the programming language Scala was new to all of us, we had to put some effort into it to understand how Scala works.
\subsection{Angular}
Angular proved to be the easiest component of the project.  However not everything was straight-forward. One tricky part was using Ajax calls to retrieve information about available bikes and displaying it. We needed a little more time than expected to use the values from \textit{.subscribe()} in some other function.
\subsection{Scala}
Scala was completely new for all of us. We were lucky enough to find a project that had some similar functionalities. Using that project as a guideline, we implemented a part of our own back-end functionality that communicates with front-end and the database. The biggest issues we encountered at using Scala was implementing WebSockets.
\subsection{Docker}
Docker appeared to be an extremely nice tool for developing applications. With docker, it has become very easy to separate the infrastructure decisions from internal application issues. Docker was pretty easy to learn and quite easy to use. The documentation appears to be more or less comprehensive and docker itself seems to be pretty user-friendly. The main issues that we had in this part were mainly connected to deploying certain containers and setting the right configurations for all of them and not to the docker logic in general.

Docker swarm mode is very useful and its logic is easy to understand. It enables fault tolerance in a very intuitive and straightforward way by replicating the services among different nodes in the cluster or even within a single node. Getting acquainted with Docker definitely was one of the best things to do during this course. 

\subsection{Future work}
Probably the most important feature that is missing is the live streaming of data. It was supposed that bike locations will change every 10 or 30 seconds in correspondence to the real ones. Another important part of the project that is missing is the login and register part for users of the application. By saying that it is missing we mean that it is not functional. The form components have been created, but they do nothing when \textit{Submit} button is pressed. Users should be able to create an account and to login. When a user is logged in, he can rent a bike and see his history of rented bikes. This history can take vary from just the day and the coordinates/address of the start and end point to a map that shows all the intermediary points recorded using live data streaming. Another missing part is the history of the bike. Due to the difficulties we encountered there was not enough time to implement this correctly. Finally, some stylistic upgrades can be done. However this is not important in the context of the functionality of the project.

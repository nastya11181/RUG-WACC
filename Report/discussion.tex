\section{Discussion}

In this chapter we review and reflect on the project.

At the beginning of the project we started with developing an UI. After that we created an nginx webserver and implemented a backend service using Play framework, but we struggled a lot to provide the connection from the front-end to the back-end. It took us quite a lot of time to figure it out how exactly to connect all the parts together. We ended up using Rest API  to connect the front-end to the back-end. We also tried to implement WebSockets. At the end we finally managed to successfully retrieve the data from the database and to present it in the UI. As already mentioned, we chose to use Angular as a framework for our UI.

It took us some time to set up the back-end with Play and Scala. Since the programming language Scala was new to all of us, we had to put some time and effort in understanding how Scala works.

\subsection{Angular}
Angular proved to be the easiest component of the project.  However not everything was straightforward. One tricky part was using Ajax calls to retrieve information about available bikes and displaying it. We needed a little more time than expected to use the values from \textit{.subscribe()} in some other function.
\subsection{Scala}
Scala was completely new to all of us. We were lucky enough to find a project that provided some similar functionalities. Using that project as a guideline, we implemented a part of our own back-end functionality that communicates with front-end and the database. The biggest issue we encountered at using Scala was implementing WebSockets because we could not find many examples online.
\subsection{Docker}
Docker appeared to be an extremely nice tool for developing applications. With Docker, it has become very easy to separate the infrastructure decisions from the internal application issues. Docker was pretty easy to learn and quite easy to use. The documentation appears to be more or less comprehensive and Docker itself seems to be very user-friendly. The main issues that we had in this part were mainly connected to deploying certain containers and figuring out the correct settings for them. For example, sbt failed to dockerize our Play application, so we had to discover an alternative way to do that. Replicating MongoDB instances and importing data in the database also caused some problems. However, the docker logic overall seemed to be very straightforward and did not cause significant issues.

Docker swarm mode that we used in our application appears to be extremely useful in providing fault tolerance and its logic is easy to understand. It enables fault tolerance in a very intuitive way by replicating the services among different nodes in the cluster or even within a single node. Getting acquainted with Docker definitely was one of the most valuable and beneficial tasks during this course.

\subsection{Future work}
Probably the most important feature that is missing is the live streaming of the data. It was supposed that the bike locations would change every 10 or 30 seconds in correspondence to the real ones. Another important part of the project that is missing is the login and register part for the users of the application. By saying that it is missing we mean that it is not functional. The form components have been created, but they do nothing when \textit{Submit} button is pressed. Users should be able to create an account and to login. When a user is logged in, he can rent a bike and see his history of rented bikes. This history can vary from just the day and the coordinates/address of the start and end points to a map that shows all the intermediary points recorded using live data streaming. Another missing part is the history of the bike. Due to the difficulties we encountered there was not enough time to implement this correctly. Finally, some stylistic upgrades can be done. However this is not important in the context of the functionality of the project, which now represents a fully connected functioning system.
